\documentclass[12pt,a4paper]{article}

\makeatletter
    \usepackage[utf8]{inputenc}
\usepackage[T1]{fontenc}
\usepackage{ucs}

\usepackage[french]{babel,varioref}

\usepackage[top=2cm, bottom=2cm, left=1.5cm, right=1.5cm]{geometry}
\usepackage{enumitem}

\usepackage{pgffor}

\usepackage{multicol}

\usepackage{makecell}

\usepackage{color}
\usepackage{hyperref}
\hypersetup{
    colorlinks,
    citecolor=black,
    filecolor=black,
    linkcolor=black,
    urlcolor=black
}

\usepackage{amsthm}

\usepackage{tcolorbox}
\tcbuselibrary{listingsutf8}

\usepackage{ifplatform}

\usepackage{ifthen}

\usepackage{macroenvsign}


% Sections numbering

\renewcommand\thesection{\arabic{section}.}
\renewcommand\thesubsection{\alph{subsection}.}
\renewcommand\thesubsubsection{\roman{subsubsection}.}


% MISC

\newtcblisting{latexex}{%
	sharp corners,%
	left=1mm, right=1mm,%
	bottom=1mm, top=1mm,%
	colupper=red!75!blue,%
	listing side text
}

\newtcbinputlisting{\inputlatexex}[2][]{%
	listing file={#2},%
	sharp corners,%
	left=1mm, right=1mm,%
	bottom=1mm, top=1mm,%
	colupper=red!75!blue,%
	listing side text
}


\newtcblisting{latexex-flat}{%
	sharp corners,%
	left=1mm, right=1mm,%
	bottom=1mm, top=1mm,%
	colupper=red!75!blue,%
}

\newtcbinputlisting{\inputlatexexflat}[2][]{%
	listing file={#2},%
	sharp corners,%
	left=1mm, right=1mm,%
	bottom=1mm, top=1mm,%
	colupper=red!75!blue,%
}


\newtcblisting{latexex-alone}{%
	sharp corners,%
	left=1mm, right=1mm,%
	bottom=1mm, top=1mm,%
	colupper=red!75!blue,%
	listing only
}

\newtcbinputlisting{\inputlatexexalone}[2][]{%
	listing file={#2},%
	sharp corners,%
	left=1mm, right=1mm,%
	bottom=1mm, top=1mm,%
	colupper=red!75!blue,%
	listing only
}


\newcommand\inputlatexexcodeafter[1]{%
	\begin{center}
		\input{#1}
	\end{center}

	\vspace{-.5em}
	
	Le rendu précédent a été obtenu via le code suivant.
	
	\inputlatexexalone{#1}
}


\newcommand\inputlatexexcodebefore[1]{%
	\inputlatexexalone{#1}
	\vspace{-.75em}
	\begin{center}
		\textit{\footnotesize Rendu du code précédent}
		
		\medskip
		
		\input{#1}
	\end{center}
}


\newcommand\env[1]{\texttt{#1}}
\newcommand\macro[1]{\env{\textbackslash{}#1}}



\setlength{\parindent}{0cm}
\setlist{noitemsep}

\theoremstyle{definition}
\newtheorem*{remark}{Remarque}

\usepackage[raggedright]{titlesec}

\titleformat{\paragraph}[hang]{\normalfont\normalsize\bfseries}{\theparagraph}{1em}{}
\titlespacing*{\paragraph}{0pt}{3.25ex plus 1ex minus .2ex}{0.5em}


\newcommand\separation{
	\medskip
	\hfill\rule{0.5\textwidth}{0.75pt}\hfill
	\medskip
}


\newcommand\extraspace{
	\vspace{0.25em}
}


\newcommand\whyprefix[2]{%
	\textbf{\prefix{#1}}-#2%
}

\newcommand\mwhyprefix[2]{%
	\texttt{#1 = #1-#2}%
}

\newcommand\prefix[1]{%
	\texttt{#1}%
}


\newcommand\inenglish{\@ifstar{\@inenglish@star}{\@inenglish@no@star}}

\newcommand\@inenglish@star[1]{%
	\emph{\og #1 \fg}%
}

\newcommand\@inenglish@no@star[1]{%
	\@inenglish@star{#1} en anglais%
}


\newcommand\ascii{\texttt{ASCII}}


% Example
\newcounter{paraexample}[subsubsection]

\newcommand\@newexample@abstract[2]{%
	\paragraph{%
		#1%
		\if\relax\detokenize{#2}\relax\else {} -- #2\fi%
	}%
}



\newcommand\newparaexample{\@ifstar{\@newparaexample@star}{\@newparaexample@no@star}}

\newcommand\@newparaexample@no@star[1]{%
	\refstepcounter{paraexample}%
	\@newexample@abstract{Exemple \theparaexample}{#1}%
}

\newcommand\@newparaexample@star[1]{%
	\@newexample@abstract{Exemple}{#1}%
}


% Change log
\newcommand\topic{\@ifstar{\@topic@star}{\@topic@no@star}}

\newcommand\@topic@no@star[1]{%
	\textbf{\textsc{#1}.}%
}

\newcommand\@topic@star[1]{%
	\textbf{\textsc{#1} :}%
}


    \usepackage{../04-cartesian-coordinates/01-cartesian-coordinates}

    \usepackage{03-vector-products}
\makeatother


% == EXTRA == %

\usepackage[f]{esvect}
\usepackage{relsize}
\usepackage{yhmath}
\usepackage{xstring}


\makeatletter
    \newcommand\pt[1]{\mathrm{#1}}

    \newcommand\@no@point[1]{%
        \IfStrEq{#1}{i}{%
            \imath%
        }{%
            \IfStrEq{#1}{j}{%
                \jmath%
            }{%
                #1
            }%
        }%
    }

    \newcommand\vect{\@ifstar{\@vect@star}{\@vect@no@star}}
    \newcommand*\@vect@star[1]{\vv*{\@no@point{#1}}}
    \newcommand*\@vect@no@star[1]{\vv{\@no@point{#1}}}
\makeatother



\begin{document}

%\section{Vecteurs}

\subsection{Produit scalaire}

Les 1\iers{} exemples utilisent une syntaxe longue mais adaptables à toutes les situations.
Voir l'exemple \ref{tnsgeo-long-dot-prod} un peu plus bas pour une écriture rapide utilisable dans certains cas.


% ---------------------- %


\newparaexample{Version classique}

\begin{latexex}
$\dotprod{\dfrac{1}{2} \vect{u}}%
         {\vect{v}}$
\end{latexex}


% ---------------------- %


\newparaexample{Version \og pédagogique mais pas écolo. \fg}

Dans l'exemple suivant l'option \prefix{b} est pour \whyprefix{b}{ullet} soit \inenglish{puce}.
Cette écriture peut être utile avec des débutants mais elle est peu pratique pour une écriture manuscrite.

\begin{latexex}
$\dotprod[b]{\dfrac{1}{2} \vect{u}}%
            {\vect{v}}$
\end{latexex}


% ---------------------- %


\newparaexample{Écriture \og universitaire \fg}

Dans l'exemple suivant l'option \prefix{p} est pour \whyprefix{p}{arenthèse} et dans \prefix{sp} le \prefix{s} est pour \whyprefix{s}{mall} soit \inenglish{petit}. On rencontre souvent cette écriture dans les cursus mathématiques universitaires.

\begin{latexex}
$\dotprod[p]{\dfrac{1}{2} \vect{u}}%
            {\vect{v}}$

$\dotprod[sp]{\dfrac{1}{2} \vect{u}}%
             {\vect{v}}$
\end{latexex}


% ---------------------- %


\newparaexample{Écriture \og à la physicienne \fg}

Dans l'exemple suivant \prefix{r} est pour \whyprefix{r}{after} soit \inenglish{chevron}. Les physiciens aiment bien cette notation.

\begin{latexex}
$\dotprod[r]{\dfrac{1}{2} \vect{u}}%
            {\vect{v}}$

$\dotprod[sr]{\dfrac{1}{2} \vect{u}}%
             {\vect{v}}$
\end{latexex}


% ---------------------- %


\newparaexample{Version courte mais restrictive} \label{tnsgeo-long-dot-prod}

Dans l'exemple suivant le préfixe \prefix{v} est pour \whyprefix{v}{ecteur}.
Notons que dans ce cas les options \prefix{sp} et \prefix{sr} n'apportent rien de nouveau.

\begin{latexex}
 $\vdotprod   {u}{v}
= \vdotprod[b]{u}{v}$
 
 $\vdotprod[r]{u}{v}
= \vdotprod[p]{u}{v}$
\end{latexex}


% ---------------------- %

%
%\newparaexample{Écriture formelle développée}
%
%Ce qui suit peut rendre service... ou pas.
%Dans l'exemple ci-dessous \prefix{exp} est pour \whyprefix{exp}{and} c'est à dire \inenglish{développer}, \prefix{c} pour \macro{cdot} et enfin \prefix{t} pour \macro{times}.
%
%\begin{latexex}
%$\dotprod[exp]{\vect{u}}{\vect{v}}$
%
%$\vdotprod[cexp]{u}{v}$
%
%$\vdotprod[texp]{u}{v}$
%\end{latexex}


% ---------------------- %


\subsection{Fiche technique}

\IDmacro{dotprod}{1}{2}

\IDoption{} la valeur par défaut est \verb+u+ pour \whyprefix{u}{sual} soit \inenglish{habituel}.  Voici les différentes valeurs possibles.

\begin{enumerate}
	\item \verb+u + : écriture habituelle avec un point.

	\item \verb+b + : écriture habituelle mais avec une puce.

	\medskip
	
	\item \verb+p + : écriture \og universitaire \fg{} avec des parenthèses extensibles.

	\item \verb+sp+ : écriture \og universitaire \fg{} avec des parenthèses non extensibles.

	\medskip
	
	\item \verb+r + : écriture \og à la physicienne \fg{} avec des chevrons extensibles.

	\item \verb+sr+ : écriture \og à la physicienne \fg{} avec des chevrons non extensibles.

%	\item \verb+exp+ : une formule développée avec un espace pour séparer les facteurs de chaque produit.
%
%	\item \verb+cexp+ : comme \verb+exp+ mais avec le symbole $\cdot$ obtenu via \macro{cdot}.
%
%	\item \verb+texp+ : comme \verb+exp+ mais avec le symbole $\times$.
\end{enumerate}

\IDarg{1} le 1\ier{} vecteur qu'il faut taper via la macro \macro{vect}.

\IDarg{2} le 2\ieme{} vecteur qu'il faut taper via la macro \macro{vect}.


\separation


\IDmacro{vdotprod}{1}{2} \hfill \mwhyprefix{v}{ector}

\IDoption{} voir les explications précédentes données pour \macro{dotprod}.

\IDarg{1} le nom du 1\ier{} vecteur sans utiliser la macro \macro{vect}.

\IDarg{2} le nom du 2\ieme{} vecteur sans utiliser la macro \macro{vect}.

\end{document}
