\documentclass[12pt,a4paper]{article}

\makeatletter
    \usepackage[utf8]{inputenc}
\usepackage[T1]{fontenc}
\usepackage{ucs}

\usepackage[french]{babel,varioref}

\usepackage[top=2cm, bottom=2cm, left=1.5cm, right=1.5cm]{geometry}
\usepackage{enumitem}

\usepackage{pgffor}

\usepackage{multicol}

\usepackage{makecell}

\usepackage{color}
\usepackage{hyperref}
\hypersetup{
    colorlinks,
    citecolor=black,
    filecolor=black,
    linkcolor=black,
    urlcolor=black
}

\usepackage{amsthm}

\usepackage{tcolorbox}
\tcbuselibrary{listingsutf8}

\usepackage{ifplatform}

\usepackage{ifthen}

\usepackage{macroenvsign}


% Sections numbering

\renewcommand\thesection{\arabic{section}.}
\renewcommand\thesubsection{\alph{subsection}.}
\renewcommand\thesubsubsection{\roman{subsubsection}.}


% MISC

\newtcblisting{latexex}{%
	sharp corners,%
	left=1mm, right=1mm,%
	bottom=1mm, top=1mm,%
	colupper=red!75!blue,%
	listing side text
}

\newtcbinputlisting{\inputlatexex}[2][]{%
	listing file={#2},%
	sharp corners,%
	left=1mm, right=1mm,%
	bottom=1mm, top=1mm,%
	colupper=red!75!blue,%
	listing side text
}


\newtcblisting{latexex-flat}{%
	sharp corners,%
	left=1mm, right=1mm,%
	bottom=1mm, top=1mm,%
	colupper=red!75!blue,%
}

\newtcbinputlisting{\inputlatexexflat}[2][]{%
	listing file={#2},%
	sharp corners,%
	left=1mm, right=1mm,%
	bottom=1mm, top=1mm,%
	colupper=red!75!blue,%
}


\newtcblisting{latexex-alone}{%
	sharp corners,%
	left=1mm, right=1mm,%
	bottom=1mm, top=1mm,%
	colupper=red!75!blue,%
	listing only
}

\newtcbinputlisting{\inputlatexexalone}[2][]{%
	listing file={#2},%
	sharp corners,%
	left=1mm, right=1mm,%
	bottom=1mm, top=1mm,%
	colupper=red!75!blue,%
	listing only
}


\newcommand\inputlatexexcodeafter[1]{%
	\begin{center}
		\input{#1}
	\end{center}

	\vspace{-.5em}
	
	Le rendu précédent a été obtenu via le code suivant.
	
	\inputlatexexalone{#1}
}


\newcommand\inputlatexexcodebefore[1]{%
	\inputlatexexalone{#1}
	\vspace{-.75em}
	\begin{center}
		\textit{\footnotesize Rendu du code précédent}
		
		\medskip
		
		\input{#1}
	\end{center}
}


\newcommand\env[1]{\texttt{#1}}
\newcommand\macro[1]{\env{\textbackslash{}#1}}



\setlength{\parindent}{0cm}
\setlist{noitemsep}

\theoremstyle{definition}
\newtheorem*{remark}{Remarque}

\usepackage[raggedright]{titlesec}

\titleformat{\paragraph}[hang]{\normalfont\normalsize\bfseries}{\theparagraph}{1em}{}
\titlespacing*{\paragraph}{0pt}{3.25ex plus 1ex minus .2ex}{0.5em}


\newcommand\separation{
	\medskip
	\hfill\rule{0.5\textwidth}{0.75pt}\hfill
	\medskip
}


\newcommand\extraspace{
	\vspace{0.25em}
}


\newcommand\whyprefix[2]{%
	\textbf{\prefix{#1}}-#2%
}

\newcommand\mwhyprefix[2]{%
	\texttt{#1 = #1-#2}%
}

\newcommand\prefix[1]{%
	\texttt{#1}%
}


\newcommand\inenglish{\@ifstar{\@inenglish@star}{\@inenglish@no@star}}

\newcommand\@inenglish@star[1]{%
	\emph{\og #1 \fg}%
}

\newcommand\@inenglish@no@star[1]{%
	\@inenglish@star{#1} en anglais%
}


\newcommand\ascii{\texttt{ASCII}}


% Example
\newcounter{paraexample}[subsubsection]

\newcommand\@newexample@abstract[2]{%
	\paragraph{%
		#1%
		\if\relax\detokenize{#2}\relax\else {} -- #2\fi%
	}%
}



\newcommand\newparaexample{\@ifstar{\@newparaexample@star}{\@newparaexample@no@star}}

\newcommand\@newparaexample@no@star[1]{%
	\refstepcounter{paraexample}%
	\@newexample@abstract{Exemple \theparaexample}{#1}%
}

\newcommand\@newparaexample@star[1]{%
	\@newexample@abstract{Exemple}{#1}%
}


% Change log
\newcommand\topic{\@ifstar{\@topic@star}{\@topic@no@star}}

\newcommand\@topic@no@star[1]{%
	\textbf{\textsc{#1}.}%
}

\newcommand\@topic@star[1]{%
	\textbf{\textsc{#1} :}%
}



    \usepackage{04-vector-2D-coli-det}
\makeatother


% == EXTRA == %

\usepackage[f]{esvect}
\usepackage{relsize}
\usepackage{yhmath}
\usepackage{xstring}


\makeatletter
    \newcommand\pt[1]{\mathrm{#1}}

    \newcommand\@no@point[1]{%
        \IfStrEq{#1}{i}{%
            \imath%
      }{%
            \IfStrEq{#1}{j}{%
                \jmath%
          }{%
                #1
          }%
      }%
  }

    \newcommand\vect{\@ifstar{\@vect@star}{\@vect@no@star}}
    \newcommand*\@vect@star[1]{\vv*{\@no@point{#1}}}
    \newcommand*\@vect@no@star[1]{\vv{\@no@point{#1}}}
\makeatother



\begin{document}

%\section{Vecteurs}

\subsection{2D -- Critère de colinéarité de deux vecteurs}

\newparaexample{Version complète}

Dans l'exemple suivant, le préfixe \prefix{coli} est pour \whyprefix{colin}{éarité} et \prefix{criteria} signifie \inenglish{critère}.

\begin{latexex}
$\colicriteria{\vect{u}}{x}{y}%
              {\vect{v}}{x'}{y'}$
ou
$\colicriteria{\vect{AB}}%
              {x_B - x_A}{y_B - y_A}%
              {\vect{CD}}%
              {x_D - x_C}{y_D - y_C}$
\end{latexex}


% ---------------------- %


\newparaexample{Rédaction raccourcie pour les vecteurs}

Dans l'exemple suivant, le préfixe \prefix{v} est pour \whyprefix{v}{ecteur}.

\begin{latexex}
$\vcolicriteria{u}{x}{y}%
               {v}{x'}{y'}$
\end{latexex}


% ---------------------- %


\newparaexample{Versions sans les vecteurs}

Dans l'exemple suivant, on utilise la valeur \verb+novec+ pour l'argument optionnel de \macro{vcolicriteria} qui par défaut est \verb+vec+ pour pour \whyprefix{vec}{teur}.
À l'usage ceci permet des copier-coller très efficaces !


\begin{latexex}
 $\vcolicriteria[novec]{u}{x}{y}%
                       {v}{x'}{y'}$
\end{latexex}


% ---------------------- %


\subsection{Fiche technique}

\IDmacro{colicriteria}{1}{6} \hfill \mwhyprefix{coli}{nearity}


\IDoption{} la valeur par défaut est \verb+vec+. Voici les deux valeurs possibles.
\begin{enumerate}
	\item \verb+vec+ : les vecteurs sont affichés si besoin.

	\item \verb+novec+ : les vecteurs ne sont jamais affichés.
\end{enumerate}


\IDarg{1} le 1\ier{} vecteur qu'il faut taper via la macro \macro{vect}.

\IDargs{2..3} les coordonnées du 1\ier{} vecteur.

\IDarg{4} le 2\ieme{} vecteur qu'il faut taper via la macro \macro{vect}.

\IDargs{5..6} les coordonnées du 2\ieme{} vecteur.


\separation


\IDmacro{vcolicriteria}{1}{6} \hfill \mwhyprefix{v}{ector}

\IDarg{1} le 1\ier{} vecteur sans utiliser la macro \macro{vect}.

\IDarg{4} le 2\ieme{} vecteur sans utiliser la macro \macro{vect}.

\medskip

Pour le reste, voir les indications données ci-dessus pour \macro{colicriteria}. 

\end{document}
