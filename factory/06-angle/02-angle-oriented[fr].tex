\documentclass[12pt,a4paper]{article}

\makeatletter
	\usepackage[utf8]{inputenc}
\usepackage[T1]{fontenc}
\usepackage{ucs}

\usepackage[french]{babel,varioref}

\usepackage[top=2cm, bottom=2cm, left=1.5cm, right=1.5cm]{geometry}
\usepackage{enumitem}

\usepackage{pgffor}

\usepackage{multicol}

\usepackage{makecell}

\usepackage{color}
\usepackage{hyperref}
\hypersetup{
    colorlinks,
    citecolor=black,
    filecolor=black,
    linkcolor=black,
    urlcolor=black
}

\usepackage{amsthm}

\usepackage{tcolorbox}
\tcbuselibrary{listingsutf8}

\usepackage{ifplatform}

\usepackage{ifthen}

\usepackage{macroenvsign}


% Sections numbering

\renewcommand\thesection{\arabic{section}.}
\renewcommand\thesubsection{\alph{subsection}.}
\renewcommand\thesubsubsection{\roman{subsubsection}.}


% MISC

\newtcblisting{latexex}{%
	sharp corners,%
	left=1mm, right=1mm,%
	bottom=1mm, top=1mm,%
	colupper=red!75!blue,%
	listing side text
}

\newtcbinputlisting{\inputlatexex}[2][]{%
	listing file={#2},%
	sharp corners,%
	left=1mm, right=1mm,%
	bottom=1mm, top=1mm,%
	colupper=red!75!blue,%
	listing side text
}


\newtcblisting{latexex-flat}{%
	sharp corners,%
	left=1mm, right=1mm,%
	bottom=1mm, top=1mm,%
	colupper=red!75!blue,%
}

\newtcbinputlisting{\inputlatexexflat}[2][]{%
	listing file={#2},%
	sharp corners,%
	left=1mm, right=1mm,%
	bottom=1mm, top=1mm,%
	colupper=red!75!blue,%
}


\newtcblisting{latexex-alone}{%
	sharp corners,%
	left=1mm, right=1mm,%
	bottom=1mm, top=1mm,%
	colupper=red!75!blue,%
	listing only
}

\newtcbinputlisting{\inputlatexexalone}[2][]{%
	listing file={#2},%
	sharp corners,%
	left=1mm, right=1mm,%
	bottom=1mm, top=1mm,%
	colupper=red!75!blue,%
	listing only
}


\newcommand\inputlatexexcodeafter[1]{%
	\begin{center}
		\input{#1}
	\end{center}

	\vspace{-.5em}
	
	Le rendu précédent a été obtenu via le code suivant.
	
	\inputlatexexalone{#1}
}


\newcommand\inputlatexexcodebefore[1]{%
	\inputlatexexalone{#1}
	\vspace{-.75em}
	\begin{center}
		\textit{\footnotesize Rendu du code précédent}
		
		\medskip
		
		\input{#1}
	\end{center}
}


\newcommand\env[1]{\texttt{#1}}
\newcommand\macro[1]{\env{\textbackslash{}#1}}



\setlength{\parindent}{0cm}
\setlist{noitemsep}

\theoremstyle{definition}
\newtheorem*{remark}{Remarque}

\usepackage[raggedright]{titlesec}

\titleformat{\paragraph}[hang]{\normalfont\normalsize\bfseries}{\theparagraph}{1em}{}
\titlespacing*{\paragraph}{0pt}{3.25ex plus 1ex minus .2ex}{0.5em}


\newcommand\separation{
	\medskip
	\hfill\rule{0.5\textwidth}{0.75pt}\hfill
	\medskip
}


\newcommand\extraspace{
	\vspace{0.25em}
}


\newcommand\whyprefix[2]{%
	\textbf{\prefix{#1}}-#2%
}

\newcommand\mwhyprefix[2]{%
	\texttt{#1 = #1-#2}%
}

\newcommand\prefix[1]{%
	\texttt{#1}%
}


\newcommand\inenglish{\@ifstar{\@inenglish@star}{\@inenglish@no@star}}

\newcommand\@inenglish@star[1]{%
	\emph{\og #1 \fg}%
}

\newcommand\@inenglish@no@star[1]{%
	\@inenglish@star{#1} en anglais%
}


\newcommand\ascii{\texttt{ASCII}}


% Example
\newcounter{paraexample}[subsubsection]

\newcommand\@newexample@abstract[2]{%
	\paragraph{%
		#1%
		\if\relax\detokenize{#2}\relax\else {} -- #2\fi%
	}%
}



\newcommand\newparaexample{\@ifstar{\@newparaexample@star}{\@newparaexample@no@star}}

\newcommand\@newparaexample@no@star[1]{%
	\refstepcounter{paraexample}%
	\@newexample@abstract{Exemple \theparaexample}{#1}%
}

\newcommand\@newparaexample@star[1]{%
	\@newexample@abstract{Exemple}{#1}%
}


% Change log
\newcommand\topic{\@ifstar{\@topic@star}{\@topic@no@star}}

\newcommand\@topic@no@star[1]{%
	\textbf{\textsc{#1}.}%
}

\newcommand\@topic@star[1]{%
	\textbf{\textsc{#1} :}%
}



	\usepackage{02-angle-oriented}
\makeatother

% == EXTRA == %

\usepackage[f]{esvect}
\usepackage{relsize}
\usepackage{yhmath}
\usepackage{xstring}


\makeatletter
    \newcommand\pt[1]{\mathrm{#1}}

	\newcommand\tnsgeo@no@point[1]{%
		\IfStrEq{#1}{i}{%
			\imath%
		}{%
			\IfStrEq{#1}{j}{%
				\jmath%
			}{%
				#1
			}%
		}%
	}

	\newcommand\vect{\@ifstar{\@vect@star}{\@vect@no@star}}
	\newcommand*\@vect@star[1]{\vv*{\tnsgeo@no@point{#1}}}
	\newcommand*\@vect@no@star[1]{\vv{\tnsgeo@no@point{#1}}}
\makeatother



\begin{document}

% \section{Géométrie}

%\section{Angles}

\subsection{Angles orientés de vecteurs}

\paragraph{Sans chapeau - Version longue}

L'option par défaut est \prefix{p} pour \whyprefix{p}{arenthèse}.
Dans \prefix{sp} le \prefix{s} est pour \whyprefix{s}{mall} soit \inenglish{petit}.
 
\begin{latexex}
$\angleorient    {\dfrac{1}{2} \vect{i}}%
                 {\vect{j}}$

$\angleorient[sp]{\dfrac{1}{2} \vect{i}}%
                 {\vect{j}}$
\end{latexex}


% ---------------------- %


\paragraph{Sans chapeau - Version courte mais restrictive}

Dans l'exemple suivant, le préfixe \prefix{v} est pour \whyprefix{v}{ecteur} qui permet de simplifier la saisie quand l'on a juste des vecteurs nommés avec des lettres
\emph{(notez que l'option \prefix{sp} n'apporte rien de nouveau)}.

\begin{latexex}
$\vangleorient    {i}{j}$ comme
$\vangleorient[sp]{i}{j}$
\end{latexex}


% ---------------------- %


\paragraph{Avec un chapeau}

Dans l'exemple suivant, \prefix{h} est pour \whyprefix{h}{at} soit \inenglish{chapeau}.
Notez au passage que \prefix{sh} produit juste des parenthèses petites mais ce choix de nom simplifie l'utilisation de la macro \emph{(c'est mieux que \prefix{hsp} par exemple)}.

\begin{latexex}
$\angleorient[h] {\dfrac{1}{2} \vect{i}}%
                 {\vect{j}}$

$\angleorient[sh]{\dfrac{1}{2} \vect{i}}%
                 {\vect{j}}$

$\vangleorient[h] {i}{j}$ comme
$\vangleorient[sh]{i}{j}$
\end{latexex}


% ---------------------- %


\subsection{Fiche technique}

\IDmacro{angleorient}{1}{2}

\IDoption{} la valeur par défaut est \verb+p+.  Voici les différentes valeurs possibles.
\begin{enumerate}
	\item \verb+p+ : écriture habituelle avec des parenthèses extensibles.

	\item \verb+sp+ : écriture habituelle avec des parenthèses non extensibles.

	\item \verb+h+ : écriture avec un chapeau et des parenthèses extensibles.

	\item \verb+sh+ : écriture avec un chapeau et des parenthèses non extensibles.
\end{enumerate}

\IDarg{1} le premier vecteur qu'il faut taper via la macro \macro{vect}.

\IDarg{2} le second vecteur qu'il faut taper via la macro \macro{vect}.


\separation


\IDmacro{vangleorient}{1}{2} \hfill \mwhyprefix{v}{ector}

\IDoption{} voir les explications précédentes données pour \macro{angleorient}.

\IDarg{1} le nom du premier vecteur sans utiliser la macro \macro{vect}.

\IDarg{2} le nom du second vecteur sans utiliser la macro \macro{vect}.

\end{document}
